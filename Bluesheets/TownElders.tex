\documentclass[blue]{Pestilence}
\begin{document}
\name{\bTownElders{}}

The town on Gorkhon River is a small, insular community with its own beliefs, notably unlike those of the neighboring Buryats, Kazakhs and Mongolians. The town is regarded with suspicion by steppe nomads and ridiculed by city folks, but a few desperate outsiders have been known to travel there to seek treatment for incurable ailments.

The town's administrative and spiritual leader is called an elder. The elder is responsible for guiding the town, healing the sick, and nominally, performing the sacrifices necessary to maintain the health of the town. No human sacrifices have been recorded for three decades, but the town has been prospering until now.

An elder must be descended from one of the town's few spiritually powerful families. They are not allowed to marry, but they must have a spiritually sensitive heir to pass on their responsibilities to. The town's current elder is rather unusual for having two children. 

Because of the ridicule the townsfolk face from the rest of the world, they tend to be hostile towards outsiders. They do not look kindly upon outsiders disrespecting their customs or breaking their long-held taboos. Such taboos include:

\begin{itemz}
	\item Drawing blood
	\item Burning bodies
	\item Desecrating dead bodies
	\item Consumption of alcohol
\end{itemz}

An elder may give permission to break some of these taboos in desperate situations. However, an elder's reputation will drop drastically if they are seen engaging in taboo activities themselves.

\end{document}
