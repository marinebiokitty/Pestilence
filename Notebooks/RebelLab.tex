\documentclass[greennotebook]{Pestilence} %% [notebook] or [greennotebook]
\begin{document}

\startnotebook{\nRebelLab{}}

\begin{page}{first}

You've always had power over the body, even without the earth spirit's help. If you listen closely enough, you can hear the rush of blood and feel the rub of muscles against bone. Perhaps this power might help you fight the plague. 

Study the inner workings of three different healthy people who are not related to you. To listen to a body, get a character's consent and then stand within one ZOC of your subject for one minute without speaking. NPC automatically refuse consent. The action is interrupted if you speak or if the target is more than one ZOC away. Once done, go to page \nbref{second}

\end{page}

\begin{page}{second}

Happy families are all alike, but every unhappy family is unhappy in its own way. Tolstoy wrote that, you believe. You could say the same about the human body. Healthy bodies at least are surprisingly similiar. Listening to them gave you a good understanding of the lines that should run through a body. Now you need to examine an unhealthy body somehow. This is a safe shelter, so you're not entirely sure where you'd find an infected person. Perhaps \cOutsider{} might have clues.

Find an infected PC and listen to their body for a minute. The infected person can be you. NPCs will refuse to consent as usual. Once done, go to page \nbref{third}.

\end{page}

\begin{page}{third}

It's all becoming clearer. Your firsthand knowledge of the body, coupled with \cOutsider{}'s anatomy textbooks, gives you an idea. You wonder if you might be able to control a body from the outside, without making any incisions. You should try it on yourself first.

Find somewhere that you consider safe. Once you've done so, open page \nbref{fourth}.

\end{page}

\begin{page}{fourth}

Your CR is reduced by 1 for 10 minutes.

It didn't go terribly wrong. You feel a little weaker, but for a moment, you could control the beating of your heart and the flow of blood in your veins. You would have been more careful if it weren't for the urgency of a plague beating down on you. You realize that you should carry more experiments just to make sure, but you can barely wait. If you can't find some way to help the infected before the army arrives, the infected will be left for dead. Sure, \textit{you} might survive, but how much will your life be worth by then?

It's time for accelerated research. You're going to try your psychic surgery techniques on someone else. Find a willing participant for your experiment. Once you do, turn to page \nbref{fifth}.

\end{page}

\begin{page}{fifth}

Oh, no! You didn't mean to do that! Your test subject is wounded.

You could try to help them, but you don't want to risk it again. Find someone who knows first aid. As your test subject is being revived, turn to page \nbref{sixth}.

If your test subject dies, you fall into a spiral of grief and guilt. If your test subject is healed through magical means or if they were healed back to consciousness while you were away, you're paralyzed by your feelings of incompetence and uselessness. In both cases, go to \nbref{seventh} if \cOutsider{} has completed \cOutsider{\their} research and needs your help.

\end{page}

\begin{page}{sixth}

At first, the first aid doesn't seem to be working. You've messed up something terribly. You're falling into panic. It's all your fault. This plague, this mess, now this death, it's all your fault, your fault, your fault...no, no, no. You did this. Only you can fix this. You sit beside your test subject and reach into them again. Clot the blood here and there. Mend the flesh. Reduce the swelling. The patient is starting to breathe normally. They're safe. You may or may not roleplay this process as the patient is being healed. 

Either way, the first aid works as normal. You may open Packet F after the patient wakes. 

You still don't know the nexus of the plague within the body. If only you knew, you would be able to move the blood here and there and nip the disease before it could kill its host. Wasn't \cOutsider{} studying the localization of the disease?

When \cOutsider{} completes their research notebook, turn to \nbref{seventh}. If \cOutsider{} is dead or if the notebook is destroyed, you are stuck. You can make no more progress.

\end{page}

\begin{page}{seventh}

That's it! That's it! \cOutsider{} is a genius! \cOutsider{\They} is describing a difficult procedure, but the notes are thorough enough that you might be able to pick it up. Hope fills you again. You can fight this.

You may now open Packet C.

\end{page}

\endnotebook

\end{document}
