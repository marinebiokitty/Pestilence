\documentclass[greennotebook]{Pestilence} %% [notebook] or [greennotebook]
\begin{document}

\startnotebook{\nOutsiderLab{}}

\begin{page}{first}

You wish to study infected blood, but you do not have your trusty blood testing kit with you. No matter, you're a scientist. A good scientist understands every method that they use. You will be able to recreate the serums and equipment required to investigate tainted blood if you have some good references.

Get three different blood samples from healthy individuals. You may take a sample from yourself if you are uninfected. Once you've gathered all samples, destroy them and go to page \nbref{second}

\end{page}

\begin{page}{second}

The results of your new test are consistent. Wonderful! Of course, if you wanted a truly reliable blood test, you would have used more than three samples, but you can't be picky during a plague. 

Now is the hard part of your research. You need to find someone who is both infected and willing to give you blood. It's far too dangerous to go outside the shelter. Moreover, the townsfolk seem to have some superstitions about drawing blood. Such nonsense, but you prefer not to be drawn and quartered by the mob outside.

You can cause an infection through unsanitary practices. You can find the dead body of an infected person. You can \textit{make} a dead body to infect someone with. The thought of inviting the very enemy you fight makes your skin crawl, but so does the thought of letting the hundreds of infected people outside die because you are too squeamish to do what needs to be done.

Draw blood from an infected PC. You may draw blood from yourself if you are infected. Consume the blood and turn to page \nbref{third}

\end{page}

\begin{page}{third}

It's no use. You can tell the blood is infected alright, but little else. The frustration is eating you inside out.

Insult someone you don't like and then vent out your frustrations more productively to an understanding person. Once you've done so, open page \nbref{fourth}.

\end{page}

\begin{page}{fourth}

You have a terribly short temper, you'll admit that. It embarrasses you to lose control so often. When you go back to the capital, you'll make sure to try some meditation. \textit{If} you go back.

No, no need to be so pessimistic. If anything, the bit of raving managed to clear your mind. Venal blood offers no clues to the nexus of the disease, but would tissue or arterial blood be better? A chill runs down your spine. Considering the pathetic materials you have, your investigation will be too invasive for a living individual. You need someone dead.

Get access to the dead body of an infected person. Perform a three minute interruptible autopsy on them. Once you do, turn to page \nbref{fifth}.

\end{page}

\begin{page}{fifth}

Hmm, decreased level of infection, but a large amount of plague specific antibodies. These antibodies couldn't help their host when they were alive, but finding them gives you an idea--an idea for a vaccine. Unfortunately, infected blood contains pathogens that you do not want to incorporate into your vaccine. Healthy blood does not have the antibodies you need. What to do, what to do.

Acquire the blood of someone who was infected and yet has somehow beaten the infection. You may draw blood from yourself if you meet the criteria. Consume the blood and afterwards, ask the blood donor out of character if they were truly infected and cured. If they were, turn to \nbref{sixth}. If they weren't, you are aware that you were cheated. Try again until you get the blood you need.

\end{page}

\begin{page}{sixth}

You've done it! You have a vaccine! Open Envelope V. You only have the resources to make this small dose. Use it wisely.

You feel like you have missed something, however. Surely you can help those already infected. You are running low on medicine and material, so the prospect of a chemical cure is slim, but how about a surgical cure? Where does the disease gather and reproduce? Can you take out its little nexus? You've realized that the pathogens die too quickly after their host dies, so you doubt you'll be able to gather much information from a dead body this time. However, you know where to look now and you can examine a living body without killing your patient.

Knock out an infected person and perform a three minute interruptible biopsy on them. You may or may not obtain their consent beforehand. NPCs are scared of you and their family will interrupt your biopsy. Once you're finished, turn to page \nbref{seventh}.

\end{page}

\begin{page}{seventh}
You have good and bad news. Good news is that you know where the disease is localized in the body! Bad news is that you can't do much else. You don't have the tools to block the function of the infected marrows without killing your patient. This is impossible. 

...unless \cRebel{} knows something. All your time with \cRebel{\them} and the town has taught you to expect the unexpected once in a while. Is \cRebel{} safe? Is \cRebel{\they} progressing on \cRebel{\their} research? Oh, please let \cRebel{\them} be alright.

Inform \cRebel{} that you know where the disease is localized. Ask \cRebel{\them} how \cRebel{\their} own research is progressing. Once you've consulted with \cRebel{\them}, you will come to accept \cRebel{}'s magic, and \cRebel{\them} performing magical feats will no longer increase your belief score. You are close. So very close.

\end{page}

\endnotebook

\end{document}
