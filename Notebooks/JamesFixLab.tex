\documentclass[greennotebook]{guildcamp4} %% [notebook] or [greennotebook]
\begin{document}

\startnotebook{\nJamesFixLab{}}

\begin{page}{first}

In order to use your equipment for your further research, you will first have to fix it. First you will need to get ahold of a \iWrench{} to open the panel on the back of your centrifuge.

Acquire a \iWrench{} and go to your \sCentrifugeBroken{}. To open the back, hold the \iWrench{} to it for an interruptible 10-count. Then open page \nbref{second}.

\end{page}

\begin{page}{second}

Great, now the back of the \sCentrifugeBroken{} is open, allowing you to get at the insides. You realize that several wires have snapped over time. You're not sure how to fix them.

But you know that \cVthree{} probably has the information for the fix on \cVthree{\their} computer. Find \cVthree{\their} computer terminal and hack into it. Then open page \nbref{third}.

\end{page}

\begin{page}{third}

Now you know how to fix the \sCentrifugeBroken{}!

Acquire some \iWires{} and hold them to your \sCentrifugeBroken{} for an interruptible 20-count. Destroy the \iWires{} item card after you do this. Then open page \nbref{fourth}.

\end{page}

\begin{page}{fourth}

Hmm... Something's still not right. You realize that you'll need to get ahold of a \iCalibrator{} to get the \sCentrifugeBroken{} working.

Acquire an \iCalibrator{} and go to your \sCentrifugeBroken{}. Hold the \iCalibrator{} to the \sCentrifugeBroken{} for an interruptible 20-count. Then open page \nbref{fifth}.

\end{page}

\begin{page}{fifth}

That did the trick! Now you just have to close the back of the \sCentrifugeBroken{} to avoid shocking yourself or causing further damage.

Take an \iWrench{} and hold it to the \sCentrifugeBroken{} for an interruptible 10-count. Then open page \nbref{sixth}.

\end{page}

\begin{page}{sixth}

Your \sCentrifuge{} is now fixed! You may now tear down the sign that says it is broken.

\end{page}

\endnotebook

\end{document}
