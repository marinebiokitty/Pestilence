\documentclass[greennotebook]{guildcamp4} %% [notebook] or [greennotebook]
\begin{document}

\startnotebook{\nSpiteNotebook{}}

\begin{page}{first}

First things first: you need to get out of this holding cell. Once you've done that, open page \nbref{second}.

\end{page}

\begin{page}{second}

Great! Now you can finish your research on \cPlead{}'s blood. But \cPlead{\they} doesn't seem to like needles, so getting a sample might prove problematic. Maybe someone else knows something about \cPlead{his} blood?

Either acquire a sample of \cPlead{}'s blood and examine it using a \iMicroscope{}, or talk to someone who knows something about \cPlead{}'s blood type. Once you've done so, open page \nbref{third}.

\end{page}

\begin{page}{third}

Incredible. \cPlead{}'s blood is Rh$_{null}$, which has only been reported in less than a hundred humans...EVER.

Could it be this fact that makes \cPlead{} immune to the Ver's stasis pods? If so, the implications are enormous.

Now you really need a sample of \cPlead{\their} blood. If you haven't done so already, do whatever it takes to get some. Then open page \nbref{fourth}.

\end{page}

\begin{page}{fourth}

Well, that wasn't easy. You really don't want to have to get another sample from \cPlead{}. 

The problem is, you need to use the \sBloodAnalyzer{} on this sample, which consumes it. Maybe you can apply your latest research techniques to make some improvements?

Examine the \sBloodAnalyzer{} for a 15-count, then open page \nbref{fifth}.

\end{page}

\begin{page}{fifth}

You can hardly believe it. From examining the \sBloodAnalyzer{}, it's obvious that your lab's blood analysis technology is more advanced than theirs.

In fact, you've recently perfected a technique, known as theranosis, whereby much less blood is required to perform a thorough analysis. You think you can make some custom modifications...

Acquire a \iWrench{} and use it on the \sBloodAnalyzer{} for a 15-count. Then open page \nbref{sixth}.

\end{page}

\begin{page}{sixth}

...and a few tweaks to the aerosolizer nozzle...

Yes! That should do it! Now let's take a look at \cPlead{}'s blood.

Analyze \cPlead{}'s blood sample using the \sBloodAnalyzer{}. For this time only, this doesn't consume the sample. Then open page \nbref{seventh}.

\end{page}

\begin{page}{seventh}

The results of the analysis are conclusive. \cPlead{}'s blood definitely has anti-Ver properties. 

You think you can make two items from \cPlead{}'s blood: a disorienting drug, and a vulnerability agent. 

This is helpful, because \cVone{} wants to turn you into a ghoul, and you need a way to hold \cVone{\them} off.

Go to the \sCentrifuge{} and hold your sample to it for an interruptible 10-count to sediment the red blood cells. (If it is broken, you can temporarily fix it using a \iWrench{} on it for a 15-count. It breaks again after you use it.)

Decant the fluid fraction into an empty \iTestTube{}. Then open page \nbref{eighth}.

\end{page}

\begin{page}{eighth}

It worked! Now you have something to help you against the Ver. However, they're still really strong, and you'll probably need to enlist the help of others to take one down.

Cross out the item name on the \iTestTube{} containing the fluid fraction of \cPlead{}'s blood and write ``\iVerDisorientationDrug{}." You can use this on any Ver within one ZoC to make them confused and disoriented (-1 CR) for 15 minutes. This consumes the item. Multiple copies of this effect do not stack, but do reset the 15-minute timer.

Cross out the item name on the \iTestTube{} containing the solid fraction of \cPlead{}'s blood and write ``\iVerVulnerabilityDrug{}." You can use this on any Ver within one ZoC to make them vulnerable to wounds and enraged (+2 CR) for 15 minutes. This consumes the item. Multiple copies of this effect do not stack, but do reset the 15-minute timer.

It is possible for a Ver to be affected by both disorientation and vulnerability at the same time.

Then open page \nbref{ninth}.

\end{page}

\begin{page}{ninth}

You can now create more \iVerDisorientationDrug{} and \iVerVulnerabilityDrug{} from additional samples of \cPlead{}'s blood by repeating the process described on page \nbref{seventh}. 

You've also learned to apply theranosis to Ver technology. For the rest of the game, using a \iWrench{} on the \sBloodAnalyzer{} for a 15-count allows the subsequent analysis of one blood sample without consuming the sample. This can only be done by you, but can be repeated.

\end{page}

\endnotebook

\end{document}
