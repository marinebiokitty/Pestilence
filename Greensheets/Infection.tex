\documentclass[green]{Pestilence}
\begin{document}
\name{\gInfection{}}

The disease that is ravaging the streets outside is unlike anything recorded in medical literature. Some believe it to be a hyper-virulent hemorrhagic fever, while others gave it more mystical names such as blood rot or earth's death. It has no incubation period; affected individuals show obvious symptoms immediately, up to and including fever, fatigue, nausea, swollen joints, purple spots and bloodshot eyes. There is no widely known vaccine or cure at present. The disease usually kills within five to twenty four hours.

The game will end in three hours, so no one who is infected during game will die during game solely from the disease. However, the army will refuse to save those who display signs of infection, and the infected will die shortly if they end the game untreated.

Infected players must have a sticker on their name badges. Their CR is reduced by 2 as long as they are infected. CR cannot go lower than 0.

NPCs can also be infected. An NPC sheet with a sticker on it denotes an infected townsperson.

Dead bodies are biohazards, especially if they belonged to infected people. The dead body of an infected person (NPC sign with an X and a sticker on it, or an abandoned name badge with a sticker) will immediately infect anyone who is within one ZOC of it without a mask. Matches are freely available, but fuel is required to burn these bodies. In general, be wary of dead bodies. Stickers and signs of infection are both hard to see from distance.

Infection has some perks if you are the vengeful type. Once infected, you may take a Spread Plague ability card and an accompanying sheet of stickers from Envelope Z. A restrained plaguebearer is not capable of walking away, but they are still capable of infecting you if you are within arm's reach of them.

\end{document}
