\documentclass[green]{guildcamp4}
\begin{document}
\name{\gHacking{}}

\emph{(This greensheet details how to attempt to break into a computer terminal.)}

Computer terminals are susceptible to being hacked as a read only state you may not delete any information thus. Hacking terminals is a delicate and often time consuming process. The difficulty of hacking a terminal depends on the terminal itself. If anyone observes you while in this process, you must tell them that you are fiddling with the terminal in an obvious attempt to break in.\\

How to hack a terminal:
\begin{enumerate}
	\item Shuffle the deck seven times.
	\item Deal out 2 cards. This is your {\em working hand}.
	\item Deal out 5 cards in a line above your working hand. This is your {\em dynamic library}.
	\item You may swap out a card in your working hand with a card in your dynamic library.
	\item Check to see if you can make a straight of the required size from the cards in your working hand and dynamic library. If not, discard your dynamic library.
	\item Repeat steps 3-5 until you have fulfilled your success condition or you run through the deck.  If you run out of deck, you must start over from step 1.
\end{enumerate}

{\bf Success Condition by Terminal Difficulty:}\\

\begin{tabular}{||r|l||}
	\hline\hline
	Terminal Difficulty	& Required hand\\
	\hline
	0	& Straight of 4\\
	1	& Straight of 5\\
	2	& Straight of 6\\
	3	& Straight of 7\\
	4	& Straight of 8\\
	\hline\hline 
\end{tabular}

\vspace{10 mm}

Aces can count as high or low, but wrapping is not allowed. (IE: Q, K, A, 2, 3 is {\bf not} a valid 5 card straight.)

You may notice that several Terminals start out impossible. This is intentional. If you attempt to hack a terminal and fail (by running out of deck), you may try again immediately (continue the session), or you may give up (end the session). If you have tried to hack a terminal twice in the same session, the third time you try to hack a terminal in the same session, you may reduce the terminal difficulty by 1. Failing twice with the new lock difficulty will allow you to reduce it further, and so on, down to a minimum of 0.

There may be other ways to reduce the difficulty of a terminal.

\end{document}
