%%%%%
%%
%% Greensheets live in this directory.  This file doubles as a
%% latex'able example greensheet.
%%
%% _template.tex serves as a bare-bones template suitable for
%% copying when starting a new sheet.
%%
%% Greensheet macros (in ../Lists/green-LIST.tex, presumably) each
%% have a file that lives here.  The argument to \name{...} probably
%% should be the macro for the given greensheet, which will generate
%% the greensheet's name as specified in green-LIST.tex.  However, you
%% can also just use \name{Some Text} if you want.
%%
%%%%%

\documentclass[green]{guildcamp4}
\begin{document}

\name{\gEvidence{}}

You have a \iVideoCamera{\MYname}, which lets you make recordings. Recordings are either pictures of objects, videos of people doing something, or videos of people holding items.
	
\begin{enum}[Creating recordings]
    \item With one hand, make a circle and hold it up to your eye, as if looking through a telescope. Tell anyone who looks at you that you are holding up a camera.
    \item Look through the circle at an object or person for a silent, interruptible five count. The activity or object must be recorded for the entire five count.
    \item If you recorded an object, write the number from its item card or sign on a blank item card.
    \item If you recorded an activity, ask the player out of game to show you the front of the ability card for what they were doing, or sign whose instructions they were following. Write the character's badge number and the ability or sign number on a blank item card.
    \item If you recorded a person holding a bulky item, or a person in possession of an item during a search, write the character's badge number and the item number on a blank item card.
    \item Place the recording in the camera's envelope.
\end{enum}

\begin{itemz}[Notes]
	\item You can only record one person at time. If you recorded an activity being performed by more than one character, choose one.
	\item The camera can only store up to 3 recordings at a time. If it is full, you must upload a recording before starting a new one. 
	\item Cameras have biometrics that only let their owners use them. You may only open a camera's envelope if you are the owner.
	\item Cameras can be transferred to other characters, but not used or destroyed. 
	\item If you try to record something and there is no corresponding item/ability/sign number, that thing or action is not admissible as evidence. Just don't create the recording.
\end{itemz}
    
Your police cruiser, the \emph{HMS Anodyne}, has a terminal which can be used to upload recordings to the onboard black box.

\begin{enum}[Uploading evidence]
	\item Connect your camera to the terminal with an audible, interruptible 5-count.
    \item Upload a recording with an audible, interruptible 15-count.
    \item Look up the Ability, Item, or Sign number on the tables attached to the terminal ({\bf A/I/S}; first column). If the recording has a Character number, also find the row with that Character number ({\bf C}; second column).
    \item Place the recording in the specified evidence envelope ({\bf E}; third column).
    \item Resolve any instructions on the evidence envelope.
\end{enum}

\begin{itemz}[Notes]
	\item Evidence held on the black box is automatically transmitted to other Frith ships within broadcast range, if any.
\end{itemz}

\end{document}
