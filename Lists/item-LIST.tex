
%%
%% This file creates the Item, ItemPacket, ItemFold, ItemEnvelope, and
%% ItemLabel datatypes, and creates macros for each.  These are for
%% various types of in-game items.
%%
%%%%%


%%%%%
%% Item macros are for normal item cards.
\DECLARESUBTYPE{Item}{TransElement}
\PRESETS{Item}{
  \FD\MYtext	{} %% longer text of item
  \FD\MYmark	{} %% possible contents of shaded ``mark'' on card
  \FD\MYbulky	{0} %% potential bulkiness
  \FD\MYcapacity{N/A} %% potential capacity
  \sd\MYlistmap	{\item\MYname\ifx\MYnumber\empty\else\ (\MYnumber)\fi}
  }


%%%%%
%% \prop
%% \unstash
%% \bulky{<number>}
%% \contain{<number>}
%%
%% \prop inside an Item macro labels the card as a prop.  \unstash
%% labels the card as unstashable.  \bulky{n} labels the card as
%% n-hands bulky.  \contain{n} labels the card with n-hands capacity.
\def\prop{%
  \append\MYmark{ ~PROP~ }}
\def\unstash{%
  \append\MYmark{ ~UNSTASHABLE~ }}
\def\bulky#1{%
  \s\MYbulky{#1}%
  \append\MYmark{\mbox{ ~\MYbulky-Hand~Bulky~ }}}
\def\contain#1{%
  \s\MYcapacity{#1}%
  \append\MYmark{\mbox{ ~\MYcapacity-Hand~Capacity~ }}}


%%%%%
%% ItemPacket macros are for item cards with an attached packet.
%% They are a subtype of Item.
\DECLARESUBTYPE{ItemPacket}{Item}
\PRESETS{ItemPacket}{
  \F\MYcontents
  }


%%%%%
%% ItemFold macros are for items represented by just a folded packet.
%% They are a subtype of ItemPacket, with the longer text and ``mark''
%% left blank, since they have no actual item card.
\DECLARESUBTYPE{ItemFold}{ItemPacket}
\PRESETS{ItemFold}{
  \s\MYmark{}
  }


%%%%%
%% ItemEnvelope macros are for items represented by just an envelope.
%% They are a subtype of ItemPacket, with the longer text and ``mark''
%% left blank, since they have no actual item card.
\DECLARESUBTYPE{ItemEnvelope}{ItemPacket}
\PRESETS{ItemEnvelope}{
  \s\MYmark{}
  }


%%%%%
%% ItemLabel macros are for small labels that would get used on
%% physreps, e.g. gun labels.  The ``mark'' is left blank, since
%% it isn't used for these.
\DECLARESUBTYPE{ItemLabel}{Item}
\PRESETS{ItemLabel}{
  \s\MYmark{}
  }


%%%%%
%% \icard[<extras>]{<name>}{<number>}{<text>}
%% \specialicard[<extras>]{<name>}{<number>}{<text>}{<mark>}
%% \itempacket[<extras>]{<name>}{<number>}{<text>}{<mark>}{<contents>}
%% \itemfold{<name>}{<number>}{<text>}{<contents>}
%% \itemenvelope{<name>}{<number>}{<text>}{<contents>}
%% \itemlabel{<name>}{<number>}{<text>}
%%
%% These are wrappers around \INSTANCE, useful for 1-shots.
%%
%% For \icard, \specialicard, and \itempacket, the optional <extras>
%% (in []'s) is for things like \unstash and \bulky{3}.  For example,
%% \icard[\prop\contain{2}]{..}{..}{..}{..} gives an item that has a
%% prop and 3-hands capacity.
%%
%% The last arg (#5) to \specialicard is for anything extra you may
%% want in the ``mark''
\newinstance{Item}{\icard[4][]}{
  \s\MYname{#2}\s\MYnumber{#3}\s\MYtext{#4}#1}
\newinstance{Item}{\specialicard[5][]}{
  \s\MYname{#2}\s\MYnumber{#3}\s\MYtext{#4}\s\MYmark{#5}#1}
\newinstance{ItemPacket}{\itempacket[6][]}{
  \s\MYname{#2}\s\MYnumber{#3}\s\MYtext{#4}\s\MYmark{#5}\s\MYcontents{#6}#1}
\newinstance{ItemFold}{\itemfold[4]}{
  \s\MYname{#1}\s\MYnumber{#2}\s\MYtext{#3}\s\MYcontents{#4}}
\newinstance{ItemEnvelope}{\itemenvelope[4]}{
  \s\MYname{#1}\s\MYnumber{#2}\s\MYtext{#3}\s\MYcontents{#4}}
\newinstance{ItemLabel}{\itemlabel[3]}{
  \s\MYname{#1}\s\MYnumber{#2}\s\MYtext{#3}}


%%%%%%%%%%%%%%%%%%%%%%%%%%%%%%%%%%%%%%%%%%%%%%%%%%%%%%%%%%%%%%%%%%

\NEW{Item}{\iTest}{
  \s\MYname	{Test Item}
  \s\MYnumber	{0000}
  \s\MYtext	{A Test Item Card}
  }

\NEW{ItemPacket}{\iTestPacket}{
  \s\MYname	{Test Item}
  \s\MYnumber	{0000}
  \s\MYtext	{A Test Item with a big red button.  Open packet if
		you press the big red button.}
  \s\MYcontents	{The item beeps at you.}
  }

\NEW{ItemFold}{\iTestFold}{
  \s\MYname	{Test Food}
  \s\MYnumber	{0000}
  \s\MYtext	{open if you eat}
  \s\MYcontents	{It tastes yummy.}
  }

\NEW{ItemEnvelope}{\iTestEnvelope}{
  \s\MYname	{Test Food}
  \s\MYnumber	{0000}
  \s\MYtext	{open if you eat}
  \s\MYcontents	{It tastes yummy.}
  }

\NEW{ItemLabel}{\iTestLabel}{
  \s\MYname	{Test Gun Label}
  \s\MYnumber	{0000}
  \s\MYtext	{Disc gun, loadable to 20 shots.}
  }

\NEW{Item}{\iWhatzit}{
  \s\MYname	{Whatzit}
  \s\MYnumber	{12345}
  \s\MYtext	{If you press it, open packet a.  If you twirl it, open
		packet b.  If you pull it, open packet c.}
  \bulky	{1}
  \s\MYsigns	{\signstrip{a}{it goes ``beep.''}
		\signstrip{b}{it goes ``whoop.''}
		\signstrip{c}{it goes ``bang.''}
		}
  \s\MYabils	{\ability{Stop Crying}{By futzing with the Whatzit, you
		can make babies stop crying.}{I make the baby stop
		crying.}
		}
  }


\NEW{Item}{\iBlankCard}{
	\s\MYname	{Blank Item}
	\s\MYnumber	{}
	\s\MYtext	{}
	\s\MYmark	{}
}

%%%%%%%%%%%%%%%%%%%%%%%%%%%%%%%%%%%%%%%%%%%%%%%%%%%%%%%%%%%%%
%Con Artist's wares

\NEW{ItemFold}{\iRealPotion}{
	\s\MYname	{Mysterious Potion}
	\s\MYnumber	{0012}
	\s\MYtext	{Smells like...melted copper? Open if consumed}
  	\s\MYcontents	{Your insides are burning and it doesn't stop. Your CR drops to 0, and you are unable to attack, assist, use weapons or use your combat abilities for the rest of the game. If you were infected, you are no longer infected and you lose your ability to spread the plague. If you weren't infected, all you know is that you are probably crippled for the rest of your life. Item is destroyed}
}

\NEW{ItemFold}{\iFakeMedicineOne}{
	\s\MYname	{Mysterious Potion}
	\s\MYnumber	{0013}
	\s\MYtext	{Smells like an armpit. Open if consumed}
  	\s\MYcontents	{Tastes like an armpit too. Item is destroyed}
}

\NEW{ItemFold}{\iFakeMedicineTwo}{
	\s\MYname	{Mysterious Potion}
	\s\MYnumber	{0014}
	\s\MYtext	{Smells like honey and plumeria. Open if consumed}
  	\s\MYcontents	{Was that cold syrup? You don't feel any different. Item is destroyed}
}

\NEW{ItemFold}{\iFakeMedicineThree}{
	\s\MYname	{Mysterious Potion}
	\s\MYnumber	{0015}
	\s\MYtext	{Smells like tar. Open if consumed}
  	\s\MYcontents	{Ugh! You feel nauseous. Lose 1 CR for 15 minutes. Item is destroyed}
}

%%%%%%%%%%%%%%%%%%%%%%%%%%%%%%%%%%%%%%%%%%%%%%%%
%%Weapons

  \NEW{Item}{\iDagger}{
  \s\MYname	{Dagger}
  \s\MYnumber	{7742}
  \s\MYtext	{A small, easy to conceal dagger. Adds +1 to your CR. Only one weapon may be wielded at a time}
  }
 
  \NEW{Item}{\iStaff}{
 	\s\MYname	{Staff}
 	\s\MYnumber	{0063}
 	\s\MYtext	{An oaken staff. Adds +1 to your CR. Only one weapon may be wielded at a time}
	\bulky	{1}
 }
 
 \NEW{Item}{\iScalpel}{
 	\s\MYname	{Scalpel}
 	\s\MYnumber	{2626}
 	\s\MYtext	{Typically used for surgery, but may be used to cut up someone's face in a pinch. Adds +1 to your CR. Only one weapon may be wielded at a time}
 }
 
 \NEW{Item}{\iRevolver}{
 	\s\MYname	{Revolver}
 	\s\MYnumber	{0579}
 	\s\MYtext	{It has ran out of bullets, unfortunately, but can still be a good blunt weapon. Adds +1 to your CR. Only one weapon may be wielded at a time}
 }
 
  \NEW{Item}{\iBoneDagger}{
 	\s\MYname	{Bone Dagger}
 	\s\MYnumber	{0579}
 	\s\MYtext	{A sacrificial dagger. More an ornament than a weapon}
 }
 
%%%%%%%%%%%%%%%%%%%%%%%%%%%%%%%%%%%%%%%%%%%
%% Resources

  \NEW{Item}{\iBlood}{
 	\s\MYname	{Blood}
 	\s\MYnumber	{8703}
 	\s\MYtext	{You need it in your body. May only be extracted by someone with the relevant ability. Mark card with check if the blood has been extracted.}
 }
 
   \NEW{Item}{\iCoins}{
 	\s\MYname	{Coins}
 	\s\MYnumber	{8703}
 	\s\MYtext	{5 copper coins}
 }
 
%%%%%%%%%%%%%%%%%%%%%%%%%%%%%%%%%%%%%%%%%%%
%% Medical Supplies 
 
  \NEW{ItemFold}{\iAlcohol}{
 	\s\MYname	{Expensive Whiskey}
 	\s\MYnumber	{8772}
 	\s\MYtext	{Alcohol. The cause of and solution to all of life's problems. Open if drunk}
 	\s\MYcontents	{This is some ridiculously strong booze. You feel warm and giddy. Roleplay your best drunk personality for 5 minutes. Item is destroyed.}
 }
 
   \NEW{ItemFold}{\iRubbingAlcohol}{
 	\s\MYname	{Rubbing Alcohol}
 	\s\MYnumber	{8772}
 	\s\MYtext	{97 percent proof rubbing alcohol. Potent sterilizer. Open if drunk}
 	\s\MYcontents	{You regret everything. Lose 1 CR for half an hour. Roleplay your most miserable drunk personality for 5 minutes. Item is destroyed.}
 }
 
    \NEW{Item}{\iMask}{
 	\s\MYname	{Plague Mask}
 	\s\MYnumber	{8772}
 	\s\MYtext	{You cannot be infected by dead bodies as long as you have this on your person.}
 }
 
    \NEW{ItemFold}{\iVaccine}{
 	\s\MYname	{Vaccine}
 	\s\MYnumber	{8772}
 	\s\MYtext	{Open if you choose to use it}
 	\s\MYcontents	{You are immune to infection for the rest of the game. Item is destroyed.}
 }
 
%Environmental

    \NEW{Item}{\iPetrol}{
 	\s\MYname	{Fuel}
 	\s\MYnumber	{8772}
 	\s\MYtext	{Can be used to burn a body from two ZOCs away. Cannot be used on living people. Why would you do that? Re-usable}
	\bulky {1}
 }
 
 \NEW{Item}{\iRock}{
 	\s\MYname	{Rock}
 	\s\MYnumber	{8773}
 	\s\MYtext	{A rock. Or a stone, depending on your definition of smooth.}
 }
 
 %Kain family macguffins
 \NEW{ItemEnvelope}{\iNotes}{
	  \s\MYname	{Research Notes on Blood Magic}
	  \s\MYnumber	{0000}
	  \s\MYtext	{Open if you have a belief score, regardless of its current value}
	  \s\MYcontents	{At first, the notes seem little more than the town's usual magical gibberish. Upon further inspection, however, you notice some startling observations. Regenerating blood? Cells that can multiply indefinitely \texit{without} losing its differentiation and turning cancerous? You have little time to examine this closely, but you need to take these notes back to the capital and study them. This could save your lab! This could save millions of people!
	  
	  The handwriting seems familiar, however. Is this \cRebel{}'s research? Should you tell \cRebel{\them} about this? What if \cRebel{\they} refuses to let you use this? Will \cRebel{\they} think you stole from \cRebel{\them}?}
  }
  

%The crucible envelope must be stuffed with 2 portions of blood  
  \NEW{ItemEnvelope}{\iCrucible}{
	  \s\MYname	{Crucible}
	  \s\MYnumber	{0000}
	  \s\MYtext	{A small, empty clay vase that is always warm to touch. You may access the contents of this envelope if you have a gamma score of two or higher.}
	  \s\MYcontents	{Your entire body flushes with heat. If you have lost blood, the crucible lets you regenerate it. You may take blood from this envelope and treat it as non-extracted blood, but you cannot have more than three portions of blood in your body at a time. 
	  
	  If you haven't lost blood, veins burst all over your body and you lose 1 CR for half an hour. However, you realize that the crucible might help you regenerate blood if you ever lose some.
	  
	  If there is no blood in this envelope, the crucible's power has been exhausted. It's simply a clay vase now.}
  }
 
 
%%%%%%%%%%%%%%%%%%%%%%%%%%%%%%%%%%%%%%%%%%%%%%%%%%%%%%%%%%%%%%%%%%



%%%%%%%%%%%%%%%%%%%%%%%%%%%%%%%%%%%%%%%%%%%%%%%%%%%%%%%%%%%%%%%%%%
