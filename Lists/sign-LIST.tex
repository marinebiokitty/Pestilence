
%%%%
%%
%% This file sets up the Sign and Label datatypes and creates Sign and
%% Label macros.
%%
%% Signs generally represent interesting parts of game area, usually
%% as things posted on walls.  Labels represent other things, often on
%% or inside envelopes, that are part of complex mechanics.
%%
%% The default value for \MYloc will inherit location from the Place
%% or Sign most immediately up the ownership tree.  Override this by
%% setting \MYloc to anything (even blank).
%%
%% Sign is for full-sized signs that would cover most of a large
%% manila envelope; SignMedium is for signs sized to half-sized manila
%% envelopes; SignSmall is for signs sized for small manila envelopes
%% (the same size as item cards).  Label, LabelMedium, and LabelSmall
%% are analagous, but they don't have a \takedownby note at the
%% bottom.  You can always use a sign or label without an envelope or
%% with a differently-sized envelope.  Choose which based on
%% visibility and content.
%%
%% SignTiny is for signs you want to be hard to find; it is small and
%% does not have a \takedownby note.  SignDot is for a very small
%% "dot" which only has a title.
%%
%% SignStrip produces a strip of paper (without a \takedownby note)
%% with labels on the outside that show on both sides if you fold it
%% in half.  These are a convenient alternative to sub-envelopes. They
%% can also be used for "s-packets" taped to walls (see
%% Extras/README-s-packets).
%%
%% LabelCover produces a label similar to the cover to a research
%% notebook.  LabelPage, likewise, produces a page.
%%
%% EOG is for full-sized end-of-game signs.
%%
%%%%%

\DECLARESUBTYPE{Sign}{Element}
\PRESETS{Sign}{
  \FD\MYloc	{\mylocation} %% real-space location
  \FD\MYtext	{} %% text of sign
  }
\POSTSETS{Sign}{
  \edef\mylocation{\MYloc}
  \protected@edef\@ownerstring{%
    \MYname%
    \ifx\mylocation\empty\else\ (\mylocation)\fi%
    }
  }
\def\mylocation{}

\def\loc#1{\rs\MYloc{#1}}

\DECLARESUBTYPE{SignMedium}{Sign}
\DECLARESUBTYPE{SignSmall}{Sign}
\DECLARESUBTYPE{SignTiny}{Sign}
\DECLARESUBTYPE{SignDot}{Sign}
\PRESETS{SignDot}{\s\MYtext{}}

\DECLARESUBTYPE{Label}{Sign}
\PRESETS{Label}{\s\MYloc{}}
\DECLARESUBTYPE{LabelMedium}{Label}
\DECLARESUBTYPE{LabelSmall}{Label}

\DECLARESUBTYPE{SignStrip}{Sign}
\DECLARESUBTYPE{LabelCover}{Label}
\DECLARESUBTYPE{LabelPage}{Label}

\DECLARESUBTYPE{EOG}{Sign}
\PRESETS{EOG}{%
  \s\MYname	{End Of Game}
  \s\MYtext	{{\bf\Huge You may not pass through here.}}
  }


%%%%%
%% \signbig[<location>]{<name>}{<text>}
%% \eog[<location>]
%%
%% \signmdeium[<location>]{<name>}{<text>}
%% \signsmall[<location>]{<name>}{<text>}
%% \signtiny[<location>]{<name>}{<text>}
%% \signdot[<location>]{<name>}
%%
%% \labelbig{<name>}{<text>}
%% \labelmedium{<name>}{<text>}
%% \labelsmall{<name>}{<text>}
%%
%% \signstrip[<location>]{<name>}{<text>}
%% \labelcover{<name>}{<text>}
%% \labelpage{<name>}{<text>}
\newinstance{Sign}{\signbig[3][\mylocation]}{
  \s\MYloc{#1}\s\MYname{#2}\s\MYtext{#3}}
\newinstance{EOG}{\eog[1][\mylocation]}{\s\MYloc{#1}}

\newinstance{SignMedium}{\signmedium[3][\mylocation]}{
  \s\MYloc{#1}\s\MYname{#2}\s\MYtext{#3}}
\newinstance{SignSmall}{\signsmall[3][\mylocation]}{
  \s\MYloc{#1}\s\MYname{#2}\s\MYtext{#3}}
\newinstance{SignTiny}{\signtiny[3][\mylocation]}{
  \s\MYloc{#1}\s\MYname{#2}\s\MYtext{#3}}
\newinstance{SignDot}{\signdot[2][\mylocation]}{
  \s\MYloc{#1}\s\MYname{#2}}

\newinstance{Label}{\labelbig[2]}{
  \s\MYname{#1}\s\MYtext{#2}}
\newinstance{LabelMedium}{\labelmedium[2]}{
  \s\MYname{#1}\s\MYtext{#2}}
\newinstance{LabelSmall}{\labelsmall[2]}{
  \s\MYname{#1}\s\MYtext{#2}}

\newinstance{SignStrip}{\signstrip[3][\mylocation]}{
  \s\MYloc{#1}\s\MYname{#2}\s\MYtext{#3}}
\newinstance{LabelCover}{\labelcover[2]}{
  \s\MYname{#1}\s\MYtext{#2}}
\newinstance{LabelPage}{\labelpage[2]}{
  \s\MYname{#1}\s\MYtext{#2}}


%%%%%
%% \sEOG{}
%% use \sEOg[\loc{<location>}]{} for EOG sign at a specific place
\NEW{EOG}{\sEOG}{
  }


%%%%%%%%%%%%%%%%%%%%%%%%%%%%%%%%%%%%%%%%%%%%%%%%%%%%%%%%%%%%%%%%%%

\NEW{Sign}{\sTest}{
  \s\MYname	{A Room}
  \s\MYloc	{10-250}
  \s\MYtext	{A lecture hall with large, sliding blackboards.}
  }

\NEW{Sign}{\sGFG}{
	\s\MYname	{The GFG}
	\s\MYloc	{}
	\s\MYtext	{The ultimate machine. The GFG. It's big and fancy and has a few input slots and a gigantic red button and a gigantic yellow lever. How could it possibly function? What goes into these slots? What does the button do? What about the lever?}
}

\NEW{Sign}{\sBridgeC}{
	\s\MYname	{The Bridge of the Chivalry}
	\s\MYloc	{}
	\s\MYtext	{A room with several consoles that detail the status of the ship}
}

\NEW{Sign}{\sEngineC}{
	\s\MYname	{The Engine Room of the Chivalry}
	\s\MYloc	{}
	\s\MYtext	{A room with several consoles that detail the status of the ship's engine}
}


\NEW{Sign}{\sStationFourFixed}{
	\s\MYname	{Operational Engine}
	\s\MYloc	{{\it Airship Station}}
	\s\MYtext	{This device powers the Chivalry, allowing the captain to control its speed and direction. The engine is running smoothly.  You may not interact with this station unless you know otherwise.}
}

\NEW{Sign}{\sStationFourDiagnosed}{
	\s\MYname	{Sabotaged Engine}
	\s\MYloc	{{\it Airship Station}}
	\s\MYtext	{This device powers the Chivalry, allowing the captain to control its speed and direction.  Sparking dangerously the engine seems to be about to fail completely. You may not interact with this station unless you know otherwise. Upon diagnosing the problem you see that you need to find a \iCircuitBoard{}. Install the \iCircuitBoard{} by staying within one ZOC of the sign for 5 minutes with the help two other people.}
}

\NEW{Sign}{\sStationFourBroken}{
	\s\MYname	{Sabotaged Engine}
	\s\MYloc	{{\it Airship Station}}
	\s\MYtext	{This device powers the Chivalry, allowing the captain to control its speed and direction.  Sparking dangerously the engine seems to be about to fail completely. You may not interact with this station unless you know otherwise. If you have the \aRepair{} ability you may lift this sign and read the one below it.}
}

\NEW{Sign}{\sStationRunning}{
	\s\MYname	{Engine}
	\s\MYloc	{{\it Airship Station}}
	\s\MYtext	{This device powers the Chivalry, allowing the captain to control its speed and direction.  The engine is running smoothly. At T+2 Hrs remove this sign.  }
}


\NEW{Sign}{\sBioLab}{
	\s\MYname	{The Biological Laboratory}
	\s\MYloc	{}
	\s\MYtext	{A room full of bits of technological components and tools all seemingly geared toward dealing with blood in some way}
}

\NEW{Sign}{\sBloodAnalyzer}{
	\s\MYname	{Blood Analyzer}
	\s\MYloc	{}
	\s\MYtext	{A machine made to analyze blood. In order to analyze the blood you have collected you must insert the blood into the analyzer and conduct a 15 count (note this is interruptible.) 
Upon successful completion destroy the blood item card.}
}

\NEW{Sign}{\sTestTubeRack}{
	\s\MYname	{Rack of Test Tubes}
	\s\MYloc	{}
	\s\MYitems	{\multi{40}{\iTestTube{}}}
	\s\MYtext	{Rack with many TestTubes, You can take ONE ONCE EVERY 30 SEC }
}

\NEW{Sign}{\sAnodyne}{
	\s\MYname	{The HMS Anodyne}
	\s\MYloc	{}
	\s\MYtext	{The {\it HMS Anodyne}, a small Frith police cruiser.}
}


\NEW{Sign}{\sAnodyneTerminal}{
	\s\MYname	{Computer Terminal}
	\s\MYloc	{}
	\s\MYtext	{A small screen, a keyboard, and a dock for connecting portable devices.}
}

\NEW{Sign}{\sAnodyneWeaponsLocker}{
	\s\MYname	{Locked Cabinet}
	\s\MYloc	{}
	\s\MYtext	{A locked cabinet. You're not sure what's inside.}
}


%%%%%%%%%%%%%%%%%%%%%%%%%%%%%%%%%%%%%%%%%%%%%%%%%%%%%%%%%%%%%%%%%%
