\documentclass[char]{Pestilence}
\begin{document}
\name{\cOutsider{}}

You've always been far too idealistic for your own good. You didn't just desire to help people; you refused to accept disease and even death as an inevitability. You've lost friends because of your unyielding conviction, brash determination and utter distaste for any fatalism. It didn't matter. Social life was a frivolity that got in the way of your mission. You studied hard, climbed to the top and had your own clinical research laboratory at the age of 28. With full control over your own research, you were poised to study and defeat the greatest enemy of humanity: death itself.

If only it were that simple. Science has always been trapped by politics, and as the head of a lab, you began to experience its thorny tangles firsthand. Infeasible. Impractical. Those were the words they threw at you and your ideas. They told you to get your head out of the clouds and focus on something practical. It makes sense, you suppose. Normal people desire "practicality", actions whose results they can see immediately. They desire politeness instead of kindness. They desire comfort instead of good. Still, their brand of practicality was defeat in your eyes. Man would not have made it to the skies if they accepted gravity as an absolute. You will never consider death an absolute, much less a necessity.

In a last ditch attempt to save your lab, you travelled to a backwater town in the steppes, following rumors of a man who could bring people from the dead. It was a mistake. Possibly the biggest mistake you made in your life.

Truth is your shepherd, but the town and its superstitions severely stretch your definition of truth. The locals seem to believe their elder is magic and that the magic only works when \cElder{\their} coffers are full. They believe in spirits, magic powders, sorcerous families and, most disturbingly, human sacrifice. Thankfully, the practice seems to have stopped a few years before you got there, but every now and then, you hear whispers of a bull being sacrificed in the steppes. You're content to let the locals practice their strange ways, as long as they don't turn to humans for sacrifice.

What you're not content with are frauds who play with human lives. The moment you stepped off the train, the townsfolk offered you immune boosters and panaceas that were nothing except mugwort and gravel. It was all in good faith, you presumed, but it infuriated you nonetheless to realize that someone was tricking them. You asked your way around town, attempting to find the source of the fake medicine to no avail. Failing that, you set up your own little temporary practice so the locals would have access to real medicine. No one took your help. In fact, the townsfolk seem even more short-tempered with you than before.

Frustrated, you decided to focus on your mission. You put on your best suit and introduced yourself as politely as you could to the local elder, \cElder{\intro}, only to be ignored. What did the elder have to hide if \cElder{\they} was truly a healer? \cElder{\Their} children were little better at first. The older one, \cApprentice{}, is a spitting image of \cApprentice{\their} obstinate \cElder{\parent} and equally silent about \cApprentice{\their} family's suspicious practices. The younger one, on the other hand, is a different story.

\cRebel{} is a little younger than you, but equally as intelligent and bright-eyed. Sure, \cRebel{\they} still talks about sorcery and faith healing as absolute truths, but \cRebel{\they} is anything but close-minded. \cRebel{} appreciates your knowledge of chemistry and cutting-edge medicine, and in turn, you came to appreciate \cRebel{\their} wisdom, logic and surprising insightfulness. Better yet, through your talks, you realized that you share the same passion, idealism and desire to help others. \cRebel{} is one of the very few people you've met who doesn't regard you as a cold, data-cranking automation. Instead, \cRebel{\they} see you for the dreamer you are. You knew you had to go back and that nothing would come out of the odd new love, but you cherished the warmth of it nevertheless.

Again, if only it were that simple. 

A week ago, you heard a knock on your door in the middle of the night. No one visited you in your dingy old motel room, especially not at night, but you presumed it was the innkeeper having problems with his heart. Instead, you were greeted by the barrel of a six-shooter, which promptly fired into your chest. You woke up the next morning, dizzy and aching, with \cRebel{} by your bed. Your chest was wrapped in bandages, and the wound underneath them pulsed with its own rhythm. You asked \cRebel{} what happened, but \cRebel{\they} only kissed your cheek and told you to rest.

The plague started that same day. You don't believe it was a coincidence. The plague is far too virulent and far too sudden in appearance to be natural. Someone just happened to try and kill one of the few legitimate doctors in town before the plague spread. Judging by how quickly \cRebel{} seems to have found you, \cRebel{\they} too might have been attacked and on alert. You're still not sure how \cRebel{} managed to treat a point-blank ballistic wound with such efficiency, but it's a question for calmer times. 

Right now, you need to stop the plague. It has claimed thousands of lives and half the town already. The town has always rejected you, and you never had any true love for it, but hell would freeze before you yielded to your old enemy and let it claim innocent lives. With the town elder's begrudging help, you established a sterile shelter for the few healthy individuals left in town. You've contacted the authorities and received news of an army coming in to level the town. 

If the shelter remains clean of infection, the army will hopefully allow the survivors to evacuate. But you're more ambitious than that. You don't want to leave behind those who are not dead yet. And perhaps, just perhaps, a new discovery might help save your dying laboratory.

\begin{itemz}[Goals]
	\item Save as many people as possible
	\item Find a way to cure the plague
	\item Protect \cRebel{} from harm and convince \cRebel{\them} to come to the capital with you for more formal studies 
	\item Find the person who tried to kill you. Perhaps they know a thing or two about the plague
	\item Expose all fraud healers in town
	\item Research the source of the elder's "magic"
\end{itemz}

\begin{itemz}[Notes]
	\item Your belief score increases by 1 every time you witness a supernatural phenomenon or accept a supernatural explanation without vocally challenging it. You may choose not to increase your belief score if someone else provides a plausible scientific explanation for the phenomenon you witnessed. High belief score might give you more insight to the supernatural, but it might not be too good for your sanity
	\item You very much prefer to be called by your title and last name. You didn't get a degree so you could be called Fyodor or, heavens forbid, \textit{Fedya}
\end{itemz}

\begin{contacts}
	\contact{\cRebel{}} - Your trusted ally and research partner. You might be developing something more than a working relationship with \cRebel{\them}
	\contact{\cElder{}} - The town elder and a fraud most likely. You're not sure how \cRebel{} has someone like \cElder{\them} as a \cElder{\parent}
	\contact{\cApprentice{}} - A superstitious, unlikable \cApprentice{\human} who disapproves of your relationship with \cRebel{}
\end{contacts}

\end{document}

