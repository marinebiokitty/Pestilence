\documentclass[char]{Pestilence}
\begin{document}
\name{\cPlaguebearer{}}

You were a spirit of the earth once. You were the warmth of soil, the taste of copper, and the scent of thistle and hay. The town nourished you with blood, and you healed its wounds in return. Now your time as a spirit is a distant memory, a painful reminder of what you have lost. Something ripped you away from the earth and trapped you in a sterile chamber nothing like your beloved home. Its cold, dead walls sapped your strength and gave you nothing in return: no wind, no water, and no blood.

That changed one night when an unfamiliar entity broke into your prison. It grabbed your heart and took more of it than anyone ever attempted to take, so much that you thought you would expire from the strain. You did not. Instead, the walls of your prison shattered from the force of the pull. You lost half of your strength, but you were free. 

You sought nourishment immediately. You dug into the wind and water. You crawled inside a thousand bulls, horses, mice and men, coursing up their veins and drinking their blood. They called you plague, pestilence and death itself, but you called it equivalent exchange. So many of them live thanks to your pain and your sacrifice. They had forgotten to thank you in blood, and you needed to remind them.

Still, regardless of how much you drank, the earth remained as cold and unwelcoming as ever. You regained bits of your powers, but you could only harm instead of heal, devour instead of nourish. You thought yourself broken and had resigned yourself to an aimless existence apart from earth before you met \cShaman{}, a human \cShaman{\human} who knew the old ways. \cShaman{\They} offered you a braid of thistle and sang songs that reminded you of your younger days. \cShaman{\Their} voice calmed your anger and restored your mind.

\cShaman{} explained to you the events from a human's point of view. You were trapped and abused by the town's elder, \cElder{\intro}, for decades. Nearly all your healing power was drained away by your captor. Random, senseless deaths are not enough to restore this power. Instead, the townsfolk must offer a spiritually powerful individual to you as they did in older times. You decided immediately that this was the best course of action. You even felt a pang of remorse for all the innocent lives you have taken. For all your belief in equivalent exchange, you might have taken more than what you gave. Once the sacrifice is made and your powers are restored, you can repair the damages and give back what you took. \cShaman{} says the town does not deserve your mercy, however. Had they hurt \cShaman{\them} too? Did the town truly deserve what you did to them? You will have to find out.

To help you with your mission, \cShaman{} took you in and fashioned you a new body from clay, which you named \cPlaguebearer{}. You liked the sound of that name. It came to you easily, as if it belonged to someone you knew.  \cShaman{} seemed taken aback, but \cShaman{\they} did not comment on it. You've known \cShaman{\them} for only a few days, but you see her almost as a \cShaman{\parent} of sorts. The human body is giving you a human mind.

You have a plan now. A number of powerful individuals are gathered in a shelter at the elder's home, thinking themselves safe from the plague. It will be a perfect oppotunity to test them and see if the town is deserving of your mercy. If the town is filled with wretched and evil people like the elder, you will feel no remorse in letting them remain in suffering. You've already infected three of the townsfolk clandestinely to see how the others would react. It will be an interesting demonstration to say the least. While the townsfolk are being tested, you will help \cShaman{} complete the ritual sacrifice before the powerful sacrifice you need can escape with the incoming army.

There are a few notable individuals in the shelter to keep an eye out for. Most worrying is the elder who imprisoned you for so many years. You don't know what tricks \cElder{\they} and \cElder{\their} children might have in store. They might even imprison you again if you're not careful. 

Most curiously, however, you feel familiar energy welled inside one of the shelter's other inhabitants, a pompous outsider doctor by the name of \cOutsider{}. \cOutsider{\They} doesn't seem to be too fond of the elder, which is a good sign, yet something about \cOutsider{\them} is wrong. Is \cOutsider{\they} a fellow spirit? Or is it your own power that you feel?

\begin{itemz}[Goals]
	\item Regain your full power
	\item Determine if the town is worth saving
	\item Avoid being recaptured at all costs
	\item Make \cShaman{} happy
	\item Find the true nature of the outsider doctor
\end{itemz}

\begin{itemz}[Notes]
	\item You are safe from enslavement as long as you remain bound to a body
	\item Your body is made of clay, but gives all appearance of a flesh body. You have no blood to give.
\end{itemz}

\begin{contacts}
	\contact{\cShaman{}} One who knows the old ways. \cShaman{\They} is so kind to you, yet \cShaman{\they} seems so sad
	\contact{\cElder{}} Your former captor. A cruel, evil \cElder{\human}, best to avoid \cElder{\them} and \cElder{\their} kin.
\end{contacts}

\end{document}
