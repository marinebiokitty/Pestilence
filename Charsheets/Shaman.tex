\documentclass[char]{Pestilence}
\begin{document}
\name{\cShaman{}}

Oh, how joyous it is to watch them pay.

You were the town's elder once. You were not allowed to wed, for you were tainted with the blood of the earth. The one child you bore would inherit your powers and remain your only family. You accepted that. In exchange, you heard the spirits of the land and spoke to them as if they were living people. The locals came to hear your advice and receive your cures. Your cures came with their own price, however--a price that all elders knew how to pay. Blood for blood. Life for life. As cold as it sounded, the blood of a bull and a human tasted the same to the earth, so you collected livestock and other offerings in exchange for saving human lives. 

Occasionally, the earth was wounded. Poisons and metal claws dug into the earth to extract its treasures, injuring it in process. Those injuries required more than blood to heal; they required a powerful soul. You never took joy in performing those sacrifices. No matter how brave they were, how wise, and how willing, you could see the fear in their eyes before you drove the knife between their ribs. Such was the burden of an elder. 

\cElder{\intro} wanted your power but refused to bear the burden. Now the earth suffers and \cElder{\their} people die in pain. For the first time, you are eager to feed the earth and plunge your knife into the beating heart of another.

You should have fled the moment you fed the bull's blood to the earth and heard no reply. For what \cElder{\they} lacked in power, \cElder{} made through cunning and will. \cElder{} promised life without death and health without harm. \cElder{\Their} feats were undeniable and \cElder{\their} words held the town captive. The town turned against you the moment they witnessed \cElder{}'s new miracles. You tried to explain how profane the "miracles" were, and how much it would hurt the earth, but they refused to listen. 

You lost your respect for the town quickly. You would have been content to retire from authority and let \cElder{} win if not for your \cPlaguebearer{\offspring}. Your poor \cPlaguebearer{} had always been too curious, too questioning, and too rebellious. \cPlaguebearer{\They} studied the new magic and quickly determined that \cElder{} had taken the spirit from the earth and bound it to \cElder{\their} will. The discovery came at a steep prize. Two new ritual sacrifices were performed on the outskirts of town barely a day later. The wounds, the locations and the state of the bodies were unmistakable; it was the work of someone who knew the rites. Your \cPlaguebearer{\offspring} took the blame for you. You pleaded and cried, but once again, the townsfolk turned away from you--you who healed their sick and tended their crops, you who bore the weight of death in your heart so they may sleep in peace, you who had nothing left except for your child. And they hanged your \cPlaguebearer{} at dawn.

You survived only because you had the good sense to escape this time. You were old enough by then. You would never have another family or another home. All you could have was revenge. For years, you waited in the steppes, searching for other spirits to aid you in your mission. You needn't have bothered. Word reached you a week ago that the town on Gorkhon was being ravaged by a plague. Judging by the speed and ferocity of it, the plague was likely the work of the poor captured earth spirit. As much as the news pleased you, it wasn't enough. The wretched elder and \cElder{\their} children were still alive. You set out to capture the earth spirit and enact a more directed revenge of your own.

You weren't surprised to find the earth spirit broken and lost. You were surprised, however, by your own tenderness. You couldn't hurt the earth child again, not after what it had gone through. Not after it remembered the name of your \cPlaguebearer{\offspring} and took it as its own. You don't know why it did so. The name must have been familiar; the earth remembers those who sang to it. You're certain the earth child is too innocent to know and use your feelings. It is too innocent in general. It wishes to heal the townsfolk and undo its righteous vengeance. You'll have to dissuade it from such silliness. Perhaps it will rethink its mercy if it sees enough cruelty at the shelter.

You still desire some directed revenge, of course, but you could easily enact it by giving the spirit what it wants. The spirit needs the sacrifice of a powerful soul to regain its powers and return to the earth. \cElder{} is obviously a candidate for sacrifice, but where is the fun in that? No, there are greater pains in the world than death, and \cElder{} will know them firsthand. \cElder{}'s elder child is spiritually indistinguishable from a common bull. \cApprentice{\They} wouldn't restore the earth's power, but no one needs to know that before you try. The younger child, however, is as powerful as your \cPlaguebearer{} had been. Life for life, indeed.

\begin{itemz}[Goals]
	\item Enact your vengeance by killing \cElder{}'s children, preferably as sacrifices
	\item Protect the innocent earth spirit from those who would enslave it again
	\item Help interested parties uncover dirt on \cElder{}'s family
	\item Prevent the earth spirit from undoing its damage. The town deserves to suffer

\end{itemz}

\begin{itemz}[Notes]
	\item The earth spirit is unkillable, but can be enslaved if it is without a body. If the earth spirit's body is destroyed, you have the ability to build it a new body
	\item \cShaman{} is not your real name. It's the name of \cElder{\intro}'s \cElderSpouse{\spouse}. It amused you to take that name. 
\end{itemz}

\begin{contacts}
	\contact{\cPlaguebearer{}} The earth spirit, a shabnak-adyr. A powerful but innocent creature instrumental to your plans
	\contact{\cElder{}} The \cElder{\human} who took everything from you. 
	\contact{\cApprentice{}} The useless \cApprentice{\offspring} of the elder.
	\contact{\cRebel{}} The elder's more precious child. Plain murder is too crude. You will see \cRebel{\them} sacrificed instead. You'll see the agony on \cElder{}'s face and rejoice in \cElder{\their} helplessness.
\end{contacts}

\end{document}
