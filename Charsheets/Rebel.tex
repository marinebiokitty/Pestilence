\documentclass[char]{Pestilence}
\begin{document}
\name{\cRebel{}}

You never meant for this to happen. You were panicked. You weren't thinking. If only you had known the consequences, you would have thought twice. Still, you fear you would have thought twice and done it anyway.

You're the younger child of the town's elder, gifted with a spiritual power greater than that of anyone in your living family. You could hear the voices of the earth since before you could walk. Your \cElder{\parent} always loved you for your gifts, but you find it difficult to reciprocate that love, for your \cElder{\parent} is little more than a trickster despite his claims of being a healer. \cElder{\Their} powers came from an enslaved spirit of the earth whose agony you felt every time you saw \cElder{\their} oaken staff. \cElder{\They} explained that \cElder{\they} had no choice and that the alternative was human sacrifice. You wanted to believe \cElder{\them}, but the last bit of your faith shattered when you found \cElder{\them} charging \cElder{\their} own people for \cElder{\their} services. Desperate folks without the penny relied on cheap, fake medicine produced who knows where, yet your \cElder{\parent} turned a blind eye to their pain. 

Infuriated, you began to delve into forbidden blood magic in hopes that you could find a way to heal without human sacrifice or a spirit's suffering. You soon realized just why blood magic is so feared: it could hurt far more easily than it could heal. You created a crucible that could regenerate blood, but that was the extent of the good you could do before you had to take a break. You needed others' help to proceed, and you couldn't exactly tell the world you had been practicing taboo magic. Your family could be killed for that. You needed someone to trust.  

You immediately asked for the help of your \cApprentice{\sibling}, \cApprentice{}, but \cApprentice{\they} refused and left town shortly afterwards. You tried to assume the best intentions. You hoped that \cApprentice{} left because \cApprentice{\they} too was disillusioned by your \cElder{\parent}'s greed, but \cApprentice{\they} returned two years later and remained just as loyal to your \cElder{\parent} as \cApprentice{\they} ever was. You'd think \cApprentice{} would have opened \cApprentice{\their} mind a little after seeing the world outside town. You had enough trouble forgiving \cApprentice{\them} for abandoning you when you needed \cApprentice{\them}. 

All this is perhaps why you were so taken by the town's newest visitor, \cOutsider{}. Sure, \cOutsider{\they} is stubborn, tactless and constantly doubting magic that you know to be real. But \cOutsider{\they} is also brave, determined and bursting with ideas on how to defeat suffering and disease. \cOutsider{\They} is the first person you've seen in this wretched town who cares for something other than their greed and endless power struggles. You could listen to \cOutsider{\them} talk about telomeres and immortal cell lines for hours. As you listened and learned, you realized that you didn't have to remain in town and continue that hopeless, dangerous research on blood magic. Modern science interested you as much as magic and traditional medicine. You were careful enough with your answers and clever enough with your questions that \cOutsider{} came to regard you more as a partner than a student. At one point, you even hoped you could go to the capital with \cOutsider{\them} to study surgery.

All those hopes were dashed when you found \cOutsider{} dead in \cOutsider{\their} room a week ago. You had feared this would happen. \cOutsider{} was far too interested in your family secrets. You tried not to reveal too much--as much as you dislike your family, you don't want them torn by an angry mob or put in jail--but \cApprentice{} chastised you daily for your "indiscretions" anyway. At one point, \cApprentice{\they} even threatened to have \cOutsider{} killed. The night when you found \cOutsider{}'s body, you had a splitting headache the likes of which you had never experienced before. It could have simply been a migraine, but coupled with the threat, you took it for a premonition. And there you were, holding \cOutsider{}'s body in your arms.

You might have gone a little insane then. Why \cOutsider{\them}? Of all people, why \cOutsider{}, the only person who wanted to do good in this town? Just for your family's blood-soaked secrets as well. You might have hated your family for a moment. You certainly stopped thinking about them. Either way, you wrapped \cOutsider{}'s body in a blanket and marched home with the blood still splattered all over your shirt. You stole your \cElder{\parent}'s staff, returned to \cOutsider{} and put your massive powers to a good use.

You truly were as powerful as they said you were. Blood oozed anew from dried wound before the flesh sewed shut. The body took a shuddering breath. All seemed fine for a moment. Unfortunately, as powerful as you were, you were untrained. You had never wielded so much power in your life. In your rush and carelessness, you broke through the restraints that bound the earth spirit to your \cElder{\parent}'s staff. The spirit broke free. The plague started the morning after.

You haven't talked much to anyone since then. It's all your fault, says a voice in your head, perhaps your own, perhaps that of ghosts. The voice grows louder as the death toll climbs. Someone stole your research on blood magic a couple of days ago, but you were so overwhelmed by panic and guilt already that you had little time to worry. For the first time, you're glad that \cOutsider{} is a bit emotionally dense. \cOutsider{\They} has recovered fast and is already attempting to find a cure to the plague. You wouldn't have expected anything else from \cOutsider{\them}. \cOutsider{\They} trusts you unconditionally with \cOutsider{\their} research and believes that you simply nursed \cOutsider{\them} to health. You hope \cOutsider{\they} keeps thinking that way. \cOutsider{} cares for others. \cOutsider{\They} would never accept you if \cOutsider{\they} found out what you did. Neither will your family.

Despite your guilt, \cOutsider{}'s energy is infectious and you have just enough will to forge ahead and make amends for your mistakes. Perhaps you will help find the cure that \cOutsider{} is so desperately seeking for. Or perhaps there will be an opportunity to pay back in full the damage you've wrought on the earth. The earth requires equivalent exchange. Blood for blood. Life for life. Perhaps your life will pay for the one you refused to give to the earth.  

\begin{itemz}[Goals]
	\item Find a way to stop the plague. If it involves your death, so be it
	\item Protect \cOutsider{} from all harm. You don't have a future with \cOutsider{\them}, but you will not let your unwitting sacrifice be in vain
	\item Protect your family secrets from \cOutsider{}. You've already hurt your family enough
	\item Keep \cOutsider{}'s resurrection a secret from everyone
	\item Recover your notes on blood magic and the Crucible. They might be useful to you yet. They are definitely dangerous in the wrong hands
	\item Convince \cApprentice{} to make peace with \cOutsider{}, or at least stop trying to kill people
\end{itemz}

\begin{itemz}[Notes]
	\item \cOutsider{} has a belief score which goes up every time \cOutsider{\they} witnesses a supernatural event. Higher this score is, more likely \cOutsider{\they} is to gain insight to your family secrets. You can prevent \cOutsider{\them} from gaining this score by providing non-supernatural explanations to supernatural phenomenons \cOutsider{\they} sees. 
\end{itemz}

\begin{contacts}
	\contact{\cOutsider{}} - You're a little in love with \cOutsider{\them}
	\contact{\cElder{}} - Your \cElder{\parent}. Loves you dearly, but is far too morally compromised for your comfort
	\contact{\cApprentice{}} - Your \cApprentice{\sibling}. You love \cApprentice{\them}, yet you fear \cApprentice{\they} was responsible for \cOutsider{}'s murder

\end{contacts}

\end{document}

