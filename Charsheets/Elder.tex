\documentclass[char]{guildcamp4}
\begin{document}
\name{\cElder{}}

You never had much respect for traditions. Blood for blood. Life for life. Equivalent exchange. All those words reeked of surrender. Such traditions only survived because generations of men and women were too weak and too complacent to challenge them. The previous elder was one such \cShaman{\human}. \cShaman{\They} accepted \cShaman{\their} obligations meekly, choosing to drive \cShaman{\their} dagger into the hearts of pleading women instead of listening to your ideas. No, come to think of it, it was not just meekness that blocked the elder's ears; it was a matter of caste as well. You belonged to a distant cadet branch of the elder's family, and while you had decent spiritual power of your own, it was never enough to make you seem anything more than a flea in the elder's eyes. 

So you took matters into your own hands. You used your years of magical research and your clever contraptions to trap the spirit of earth within an oaken staff. An extraordinary yet loathsome creature. Shivers ran down your spine as you tapped into its power. How much blood had to be spilled on the account of this creature? You only knew one thing for certain: none would be spilled for it ever again.

The townsfolk agreed with you. They worshipped your bloodless miracles and turned against the traditions the moment they saw you cure a smallpox victim with the touch of your hands. Of course, there are always a few obstinate fellows clinging to cruelty simply because it is tradition. \cPlaguebearer{}, the former elder's \cPlaguebearer{\offspring}, was one of those people. \cPlaguebearer{\Their} \cShaman{\parent} had the wisdom to step down when time came, but the \cPlaguebearer{\kid} never had \cPlaguebearer{\their} turn and \cPlaguebearer{\they} wanted it so terribly. You showed \cPlaguebearer{\them} just how terrible the power \cPlaguebearer{\they} sought was. As much as you disliked sacrifices, you had to know your enemy to defeat it. You killed your first two men--both supporters of the previous elder--the way other elders killed thousands. You hadn't expected the townsfolk to react as violently as they did--you certainly hadn't expected them to kill \cPlaguebearer{} in an act of frenzied mob justice--but you felt little grief for \cPlaguebearer{\their} death. Let this sacrifice be the cost of your peaceful reign. Let the remaining degenerates see what they are asking to return to.

You've continued to eschew traditions even after your ascension to power. Like many other elders before you, you were supposed to bear a child with a \cElderSpouse{\human} approved by the rest of the great families. You had the child as planned, but unlike the others, you fell in love with the \cElderSpouse{\human} as well. The first time you sat beside \cElderSpouse{\intro} with \cApprentice{} in your arms, you realized that all you truly wanted was that little family. You asked \cElderSpouse{\them} to marry you, but \cElderSpouse{\they} asked you to wait until after the birth of your second child. There was never a need for such lies. \cElderSpouse{} left for the capital the day after \cRebel{} was born, leaving nothing but a note. \textit{I have borne you the child you needed,} it said. \textit{I hope to earth below that \cRebel{\they} grows to be more compassionate than you.} In the end, the traditions you destroyed avenged themselves. The bitterness of it is almost sublime.

You are so tired nowadays. The steppe winds shake your bones, and you've taken to drinking every now and then to stave off the chill. You are starting to realize that anything without a cost is quickly taken for granted. As immense as the earth spirit's powers are, its binds cannot stand too much strain. You decided to limit your services and make some money for a change. You desire some good whiskey in your old age, and your children will need more than a wretched staff and a spellbook if they are to survive in the world. Especially \cApprentice{}. The poor child is too spiritually weak to make good use of your powers, regardless of how much \cApprentice{\they} wants it. \cApprentice{\They} doesn't understand the burden, the hard decisions and the false love that come with being an elder. You're not too happy to saddle \cRebel{} with these responsibilities either, but at very least, \cRebel{\they} has the aptitude for it, if not the temperament.

Dear spirits, that temperament. That temperament was likely the cause of this entire plague business. \cRebel{} did grow up to be what most would call "compassionate." You, on the other hand, would call \cRebel{\them} delicate and overly idealistic. \cRebel{} detests your pragmatism and wants nothing to do with you. \cRebel{\They} even asked you to release the earth spirit at one point. And now the earth spirit is free. The first morning of the outbreak, you found your powerless, empty staff lying at your door, spotted with flakes of dried blood. The signs were unmistakable. Someone had attempted a resurrection with it and lost control of the spirit in process. Only \cRebel{} has enough power to pull such magnitude of power from your staff, and only \cRebel{\they} is foolish enough to even try. \cRebel{\Their} compassion must have gotten the better of \cRebel{\them} again. 

You immediately decided that word will never get out on how the plague started. You must retain your control of the town and guard your family's secrets, especially from that nosy little outsider, \cOutsider{}. That little pest is so neurotic and so entrenched in \cOutsider{\their} city science that any true magic will likely make \cOutsider{\them} reach for a torch. Worse still, the pest doesn't know \cOutsider{\their} place. Your reputation in the town has been dropping steadily due to your inability to stop the plague, and \cOutsider{} was all too happy to capitalize on it. You agreed to help \cOutsider{\them} establish the shelter in your home for the common good, but that is all the power you are willing to concede to an outsider. You dread to think of what the paranoid doctor might do if \cOutsider{\they} wields too much power. Your worst fear is that the town will find the truth and rally behind \cOutsider{\them} to kill your child like they killed the previous elder's \cPlaguebearer{\offspring}. You will never allow that. If push comes to shove, you are willing to die in \cRebel{}'s stead. 

However, you don't want to do so when there is a far better option. Whoever the earth spirit brought to life should now be a nexus of healing powers. They cannot tap into it unless they are versed in local magic, but you can if you release that energy. The earth spirit will return for their lost powers, and that will be your chance to capture it again. You dislike sacrifice, but this is not sacrifice as much as collecting a debt. Someone's borrowed time has caused a thousand deaths already. That time will soon be over, as will the spirit's freedom. 

You have no other options. Capturing the earth spirit without restoring its healing powers first would be a waste; you don't need a staff capable of only causing the plague. Letting it go and relying on its goodwill means a return to the barbaric traditions you worked so hard to erase. The city doctor seems to be trying \cOutsider{\their} own method, but \cOutsider{\their} success would spell the end of your leadership. Considering what the town did after the last change in leadership, you are not at all eager to let that happen. No, it will be you who saves the town and heals the sick. You are their elder. Let no one forget that.

\begin{itemz}[Goals]
	\item Find whoever was resurrected and sacrifice them to restore the earth's healing power
	\item Recapture the earth spirit once it has healing powers
	\item Protect your children
	\item Make sure the town listens to you, not the outsider
	\item Protect your family secrets from \cOutsider{}
	\item Make sure to have a worthy successor in case something happens
\end{itemz} 

\begin{itemz}[Notes]
	\item Earth's power can only be restored through your ritual sacrifice ability
	\item Make sure to fill out the will in your room. You don't want to leave your children squabbling over your inheritance in case something happens to you
	\item In addition to the items on your person, you have a plague mask and a sacrificial bone dagger somewhere in your house
\end{itemz}

\begin{contacts}
	\contact{\cApprentice{}} - your dutiful older child. Has little spiritual power. You wish \cApprentice{\they} would pick up interests other than magic  
	\contact{\cRebel{}} - your reckless, idealistic younger child. For all \cRebel{\their} mistakes, \cRebel{\they} is still your blood 
	\contact{\cOutsider{}} - an outsider who thinks \cOutsider{\they} owns the town. You will let \cOutsider{\them} know who is in charge soon enough
\end{contacts}

\end{document}

