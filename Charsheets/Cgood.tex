\documentclass[char]{guildcamp4}
\begin{document}
\name{\cCgood{}}

Almost since the day you were hatched, you've been told how exceptional your bloodline is. Your parents, and their parents before them, distinguished themselves as the finest police officers to ever patrol Frith space by going above and beyond the call of duty time and time again. Your forebears were responsible for some of the most dramatic arrests and busts in recent history, whether by connecting seemingly disparate pieces of evidence to uncover serial killers, tirelessly tracking movement patterns to bring about the downfall of drug trafficking rings, or going undercover and foiling terrorist plots before they could be brought to fruition; your ancestors always seem to be in the right place at the right time and able to do the right thing. Some even believe that your genetic heritage carries an inexplicable sixth sense for detecting injustice, which is, of course, an absurd notion. But sometimes, alarm bells will suddenly go off in your head, you'll feel like something isn't quite right, and that's when you'll notice something crucial that nobody else did. 

You've just graduated from the Academy with top honors, the best in your class, and can't wait to apply your deductive reasoning abilities to real-world problems instead of textbook scenarios. Your new partner and mentor, \cCbad{}, doesn't quite seem to share your enthusiasm, but that's only understandable; \cCbad{\their} bloodline was bred for military service on the front lines of war, not patrolling interior space. The Frith have recently shifted focus from military expansion to peaceful negotiation with their neighbors, which has led to many decommissioned soldiers like \cCbad{} being reassigned to police duty. Still, \cCbad{\they}'s already been on the force for a couple years, so you could probably learn a lot from \cCbad{\them}. However, you wonder what happened to \cCbad{\their} previous partner, and why \cCbad{\they} seems reluctant to talk about it.

Then, there's the question of \cPilot{}, a convict who served time for trespassing on government property, and who is currently out on parole. This morning, your sergeant pulled you aside and mentioned that \cPilot{} would be piloting your patrol ship today, as part of a rehabilitation program. Apparently, keeping ex-cons busy with gainful employment is one of the few things that actually work when trying to reintegrate them back into society. Placing them under the watchful eyes of police officers doesn't seem to hurt, either. Anyway, something about \cPilot{}'s record doesn't quite seem to add up. For one, it's completely devoid of any specific details. Which government property was \cPilot{\they} trespassing on? What was \cPilot{\they} doing there? And do \cPilot{\their} piloting and navigation skills have anything to do with it?

Today, your ship has been assigned to patrol the border near Ver space. Just now, you happened to be glancing at the long-range scanners when something caught your eye. It was a short blip, so ephemeral that it could have just been chalked up to background noise or a dust cloud. But you decided to redirect more power to the scanners, and they revealed a large Ver ship, headed for Ver space. By itself, this wouldn't be anything out of the ordinary; Human and Ver ships routinely traverse Frith space for travel, diplomacy, and trade. However, this particular ship's identification beacon was apparently inactive until you targeted it with a boosted long-range scan, and only then did it begin broadcasting designation \textit{TIS Chivalry}. In addition, this ship's class and model wasn't in any of your textbooks. You wasted no time in hailing them, calling for a customs inspection, and initiating docking procedures, which should complete in 15-20 minutes.

Today is shaping up to be anything but ordinary.

\begin{itemz}[Goals]
	\item Dock with and inspect \textit{TIS Chivalry}.
	\item Gather proof of illegal activity, if any.
	\item Report findings to headquarters.
	\item Learn what you can about being a police officer from \cCbad{}.
	\item Prove yourself as a competent officer and partner to \cCbad{}.
	\item Find out what happened to \cCbad{}'s previous partner.
	\item Make sure \cPilot{} doesn't violate \cPilot{\their} parole. 
	\item Find out more about \cPilot{}'s crime.
\end{itemz}

\begin{itemz}[Notes]
	\item 
\end{itemz}

\begin{contacts}
	%\contact{\cTest{}} <- This is the format for contacts 
	\contact{\cTest{}}
\end{contacts}

\end{document}
