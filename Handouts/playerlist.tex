%% Use [landscapeplayerlist] for landscape orientation (the page is
%% wider than it is tall) or [portraitplayerlist] for normal portrait
%% orientation.  [playerlist] will behave like [landscapeplayerlist].
\documentclass[playerlist]{guildcamp1} %% make sure name of class is correct
\begin{document}


%%%%%
%% Use either \production or \compendium, which are based on settings
%% in the .cls file.  Comment this out entirely if you are using a
%% more custom playerlist below.
\production


%%%%%
%% To specify a custom playerlist, uncomment and edit the commands
%% below, including the \begin{playerlist} (with column declarations),
%% the font size and style settings, the playerlist contents, and the
%% \end{playerlist}.
%%
%% Add columns with \column{<heading>}{<field>}, where <heading> is
%% the title of the column (displayed at the top of the playerlist)
%% and <field> is the character field that stores the value for each
%% character.  Declare and set <field> in Lists/char-LIST.tex and
%% Lists/gm-LIST.tex.
%%
%% A common column in long games with detailed playerlists is one that
%% defines the public position or role for each character within their
%% most public group, e.g. political office, military rank, corporate
%% position, or a simple title like "famous actress."  The example
%% below includes a Role column that uses the \MYrole character field.
%%
%% You might also want to remove columns; for example, if your game
%% doesn't involve player rooms, you might leave out Address.
%%
%% Each column's default width is approximately the width of the text
%% area divided by the number of columns.  You can change this for a
%% column using the optional argument in \column[<factor>]..., where
%% the column's width will be the default width times <factor>.  For
%% example, \column[.5] will be half-width, while \column[1.5] will be
%% 50% wider.  The default width is not dynamically recalculated, so
%% your column widths may total to a value greater or less than the
%% actual text area width.
%\begin{playerlist}{
%  \column{Character}{\MYname}
%  \column{Role}{\MYrole}
%  \column{Player}{\MYplayer}
%  \column{Email}{\MYemail}
%  \column{Address}{\MYaddress}
%  \column{Phone}{\MYphone}
%  \column{Notes}{\MYnote}
%  }

%% If the orientation is landscape, probably use \footnotesize.  If it
%% is portrait, probably use \tiny.
%  \footnotesize

%% \sffamily sets the font to sans-serif, which is usually both more
%% compact and easier to read at small sizes.  \slshape is just for
%% aesthetic reasons.
%  \sffamily\slshape

%% You could just use \charactergrouping or \playergrouping
%% (defined in the .cls file).  If not, see below.
%  \charactergrouping

%% Groupings of entries in the playerlist are each specified by a
%% \mark followed by one or more of \group, \groupasis, \allchars,
%% \allplayers, and \allgms.
%%
%% \mark{<text>} labels a group.
%%
%% \group{<string>} creates an alphabetically sorted group of
%% characters for which \MYgroupstr is <string> (see
%% Lists/char-LIST.tex).
%%
%% \groupasis{<string>} creates the same, except the group is in the
%% order the characters were created in Lists/char-LIST.tex.
%%
%% \allchars, \allplayers, and \allgms are all the characters,
%% players, or GMs sorted alphabetically.
%  \mark{GMs}\allgms
%  \mark{Air Force}\groupasis{airforce} %% created in rank order
%  \mark{Navy}\groupasis{navy} %% created in rank order
%  \mark{Celebrities}\group{celeb}
%  \mark{Email Parts}\group{email}

%\end{playerlist}



%%%%%
%% If the playerlist is only one page, you might want a map on the
%% reverse side
%%%%%


%%%%%
%% a more generic header
%\lhead{\small\centering\gamename\hfil\gamedate\\[\fill]}%


%%%%%
%% \clippedmap may be best if playerlist is in landscape mode
%\usemap{\clippedmap}{
%  %% trickery for use inside tabular-like env
%  \MAP{Place}{\etoss{\?\MYname\\\?\MYwhere}}%
%  \MAP{Sign}{\etoss{\?\MYname\\\?\MYloc}}%
%  %\mapgrid %% comment out \mapgrid when you're done placing labels
%  \bfseries
%  \rlabel(.85,.535){45}{}{Lookout Point\\ 66-5}
%  \clabel(.693,.355){-45}{}{The Pit}
%  \clabel(.305,.11){}{1.5}{Gates of Hell\\ MIT}
%  \clabel(.81,.377){-90}{}{Launching Bay 9}
%  \llabel(.3,.52){}{}{\sTest{}}
%  }


%%%%%
%% \landscapemap will give a landscape map when playerlist is in
%% portrait mode
%\usemap{\landscapemap}{
%  %% trickery for use inside tabular-like env
%  \MAP{Place}{\etoss{\?\MYname\\\?\MYwhere}}%
%  \MAP{Sign}{\etoss{\?\MYname\\\?\MYloc}}%
%  %\mapgrid %% comment out \mapgrid when you're done placing labels
%  \bfseries
%  \rlabel(.465,.85){45}{}{Lookout Point\\ 66-5}
%  \clabel(.645,.693){45}{}{The Pit}
%  \clabel(.89,.305){90}{1.5}{Gates of Hell\\ MIT}
%  \clabel(.623,.81){}{}{Launching Bay 9}
%  \llabel(.48,.3){}{}{\sTest{}}
%  }


%%%%%
%% \portraitmap will give a portrait map when playerlist is in
%% portrait mode
%\usemap{\portraitmap}{
%  %% trickery for use inside tabular-like env
%  \MAP{Place}{\etoss{\?\MYname\\\?\MYwhere}}%
%  \MAP{Sign}{\etoss{\?\MYname\\\?\MYloc}}%
%  %\mapgrid %% comment out \mapgrid when you're done placing labels
%  \bfseries
%  \rlabel(.85,.505){45}{}{Lookout Point\\ 66-5}
%  \clabel(.693,.405){-45}{}{The Pit}
%  \clabel(.305,.27){}{1.5}{Gates of Hell\\ MIT}
%  \clabel(.81,.415){-90}{}{Launching Bay 9}
%  \llabel(.3,.5){}{}{\sTest{}}
%  }


\end{document}
